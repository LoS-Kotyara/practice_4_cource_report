\subsection{Разработка программной части комплекса}

\subsubsection{Операционная система Manjaro}
В качестве операционной системы для управления Raspberry Pi был выбран дистрибутив Manjaro.

Manjaro является дистрибутивом Linux, основанным на Arch Linux, который позиционируется как открытая и лёгкая операционная система. Также Arch Linux и Manjaro обладают подробной документацией.

В Manjaro по-умолчанию используется ALSA в качестве звуковой подсистемы, поэтому для записи звуковых потоков используется утилита arecord, а для воспроизведения --- aplay.

Manjaro использует пакетный менеджер pacman, упрощающий управление пакетами. Для поддержания пакетов в актуальном состоянии, pacman синхронизируется с базами данных пакетов Manjaro посредством зеркал. Также возможна и установка пакетов ``вручную'' из исходных кодов.

Для начала разработки, необходимо установить все необходимые пакеты, такие как gcc, предоставляющую компиляторы языков C и C++, make, предоставляющую сборку утилит из исходных кодов, sudo, упрощающую выполнение команд от имени администратора, и других пакетов. Для установки этой группы пакетов, необходимо выполнить следующую bash-команду:

\begin{lstlisting}[style=ES6, language=bash]
  sudo pacman -Syy base-devel
\end{lstlisting}

При установке пакетов произойдёт полная синхронизация баз данных пакетов, так как был указан параметр ``yy''.

\subsubsection{Node.JS}

В качестве основного языка разработки комплекса был выбран язык JavaScript (JS). Для взаимодействия с операционной системой используется среда выполнения JS --- NodeJS (Node). Чтобы иметь возможность выполнять Node-скрипты, необходимо предварительно установить пакет `nodejs' с помощью менеджера пакетов pacman, либо с помощью менеджера версий nvm. Ввиду своей простоты, была произведена установка с помощью pacman.

\begin{lstlisting}[style=ES6, language=bash]
  sudo pacman -S nodejs
\end{lstlisting}

Вместе с установкой Node, производится установка Node Package Manager (npm), с помощью которого можно управлять Node-проектами и устанавливать Node-пакеты.

Для создания нового проекта используется npm. Для инициализации проекта, необходимо выполнить команду `npm init -y', которая создаст в текущей директории файл package.json, в котором хранится вся информация о проекте: название, автор, текущая версия, команды управления, используемые пакеты, или зависимости, и другая информация.

\subsubsection{Пакеты для управления алресными светодиодными лентами}

Для управления адресными светодиодными лентами с помощью Node, наибольшей популярностью обладают 3 пакета~\cite{npm}:

\begin{itemize}
  \item node-pixel;
  \item rpi-ws281x;
  \item rpi-ws281x-native.
\end{itemize}

Node-pixel является наиболее популярным пакетом, но он также является самым ``тяжёлым'' пакетом и предполагает использование какой-либо другой платы, например, Arduino Nano, с помощью которой происходит управление светодиодной лентой. В этом случае Raspberry Pi выполняет роль интерпретатора языка Node, а плата, принимая полученные по I2C-шине данные, управляет светодиодной лентой. Данный пакет не был использован в проекте, так как его возможности излишни.

Rpi-ws281x и rpi-ws281x-native являются пакетами, близкими по популярности. Они предоставляют возможность прямого управления светодиодной лентой с RPI, без использования каких-либо промежуточных контроллеров.

При разработке программной составляющей проекта был использован пакет rpi-ws281x-native, ввиду более подробной документации и использования при выполнении нативных привязок GPIO.

\subsubsection{Создание дочерних процессов}

Запуск системных утилит с помощью Node можно произвести с помощью запуска дочерних процессов.

Node обладает встроенным модулем `child\_process', предоставляющий возможность запуска, завершения и управления дочерними процессами. Существует 4 способа запуска дочерних процесса~\cite{node}:

\begin{itemize}
  \item child\_process.exec() --- exec;
  \item child\_process.execFile() --- execFile;
  \item child\_process.fork() --- fork;
  \item child\_process.spawn() --- spawn.
\end{itemize}

Exec и execFile используются, когда необходим запуск команды или файла, содержащего команды, или исполняемого файла, без возможности управления его стандартными потоками. В этом случае создаётся отдельный процесс, выполняемый, пока на одном из стандартных потоков не появятся данные. Например, эти команды можно использовать для запуска какой-либо системной службы или получения информации о файле.

Spawn используется для запуска команды с некоторыми параметрами запуска. В этом случае возможно управление стандартными потоками ввода, вывода и ошибок. Также, процесс, запущенный с помощью spawn будет активен, пока не получит терминальный сигнал от родительского процесса или от операционной системы.

Fork является подвидом spawn, предоставляющим возможность запуска нового Node процесса.

Для запуска процессов, выполняющих команды aplay и arecord, был использован способ spawn.

Например, используя скрипт, представленный в листинге~\ref{lst:node__spawn_pipe}, можно запустить процессы, выполняющие запись и воспроизведение звука, перенаправив выходной стандартный поток процесса записи во входной стандартный поток процесса воспроизведения.

\begin{lstlisting}[style=ES6, caption={Пример запуска процессов и перенаправления потоков процессов}, label={lst:node__spawn_pipe}]
const arecordProcess = spawn('arecord', [
    '-t', 'raw'
]);

const aplayProcess = spawn('aplay', ['-',
  '-t', 'raw'
]);

arecordProcess.stdout.pipe(aplayProcess.stdin);
\end{lstlisting}

\subsubsection{Управление адресной светодиодной лентой}

Для управления адресной светодиодной лентой используется пакет rpi-ws281x-native, который предоставляет нативные привязки GPIO для более эффективного управления светодиодными лентами.

При инициализации объекта управления светодиодной лентой, необходимо указать следующие параметры:

\begin{itemize}
  \item count --- количество светодиодов в ленте;
  \item gpio --- выход GPIO, к которому подключена лента. По-умолчанию управление происходит с помощью GPIO18;
  \item invert --- изменение выходного сигнала. Применяется, если используется логический преобразователь уровней. По-умолчанию значение false;
  \item brightness --- яркость, применяемая ко всем светодиодам в ленте. По-умолчанию максимальна, 255;
  \item stripType --- тип светодиодов в ленте. По-умолчанию WS2812.
\end{itemize}

Данные о ленте хранятся в виде массива чисел, где каждому элементу массива соответствует определённый светодиод. Обращение к светодиодам происходит по индексам.

Для отправления необходимого состояния на светодиодную ленту, используется метод render().

Перед тем, как завершить работу с лентой, необходимо очистить данные, хранящиеся на контроллерах её светодиодов. Это можно сделать с помощью метода reset(), который очистит данные на контроллерах светодиодов, и finalize(), который отключает драйвер управления светодиодной лентой и высвобождает ресурсы.

Например, в листинге~\ref{lst:node__rainbow} представлен скрипт, который запускает анимацию переливания цветов для всей светодиодной ленты.

\lstinputlisting[style=ES6, caption={Пример скрипта, запускающего анимацию переливания цветов на светодиодной ленте}, label={lst:node__rainbow}, linerange={19-29, 33-48}, consecutivenumbers=true]{assets/listings/practical/rainbow.js}

В случае, если необходимо, чтобы цвета выходили из центра светодиодной ленты и расходились в стороны, тогда скрипт будет выглядеть, как показано в листинге~\ref{lst:node__shift}. Предполагается, что число светодиодов в ленте чётное.

\lstinputlisting[style=ES6, caption={Пример функции, запускающего анимацию выхода цвета из центра светодиодной ленты в стороны}, label={lst:node__shift}, linerange={23-39}, consecutivenumbers=true]{assets/listings/practical/LED.js}

В этой функции на вход поступают массив, описывающий светодиоды в ленте, и новый цвет, который необходимо отобразить начиная с центра. Сначала находится индекс центрального элемента, затем смещаются на один влево/вправо цвета светодиодов, и добавляется новый цвет. Возвращаемым значением является новое состояние светодиодной ленты.

Для отображения полученного состояния на светодиодной ленте, можно воспользоваться функцией, предложенной в листинге~\ref{lst:node__render}.

\lstinputlisting[style=ES6, caption={Пример функции отображения нового состояния светодиодной ленты}, label={lst:node__render}, linerange={84-89}, consecutivenumbers=true]{assets/listings/practical/LED.js}
