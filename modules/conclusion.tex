\conclusion

В наши дни невозможно представить себе жизнь без искусственных источников света. Эти источники света применяются повсеместно, как для простого освещения помещений, так и для придания окружению чего-то особенного.

Проекты, связанные с концепцией ``умный дом'', являются одними из наиболее молодых и популярных направлений разработки. Одну из важнейших ролей в таких проектах является создание и использование искусственных источников света.

В ходе преддипломной практики были получены знания о способах записи, обработки и воспроизведения звуковых потоков, способы модуляции сигналов, а так же были изучены особенности работы с одноплатными компьютерами Raspberry Pi. Для достижения цели данной работы был произведён анализ предметной области, и выполнены следующие задачи:

\begin{itemize}
  \item изучены и проанализированы характеристики и особенности работы светодиодов, подходящих для реализации поставленной цели;
  \item изучены способы работы со звуковыми потоками;
  \item изучены и проанализированы характеристик платформ, подходящих для реализации поставленной задачи;
  \item изучены особенности работы с одноплатными компьютерами Raspberry Pi;
  \item разработано решение, позволяющее управлять адресными светодиодными лентами с помощью звука.
\end{itemize}