\subsection{JavaScript}

Ввиду того, что manjaro является одним из дистрибутивов Linux, на нём можно выполнять все команды, которые возможны в Linux. Также для разработки возможно использование различных языков программирования, таких как C, C++, Python, JavaScript (JS) и других. 

Javascript - это легковесный интерпретируемый динамический язык программирования, который применяется при разработке сценариев веб-страниц, которые обеспечивают интерактивность сайтам, и при разработке серверных или кроссплатформенных настольных приложений. Является одной из реализаций спецификации ECMAScript~\cite{JS}.

Для исполнения кода вне браузера, применяется Node.js. Node.js основан на движке Javascript V8, написанный на C и C++, который позволяет транслировать JS-код в машинный код~\cite{node}.

Одним из достоинств языка является наличие механизма асинхронного ввода-вывода, а именно использование неблокирующих операций ввода-вывода. Это значит, что главный поток не будет блокироваться операциями ввода-вывода и сервер будет продолжать обрабатывать запросы~\cite{node}. Это возможно из-за того, что в V8, а, следовательно, Node.js, используется цикл событий, который ожидает прибытия и проводит рассылку событий и выполняет операции только когда произошло определённое событие.

npm - это менеджер пакетов, позволяющий разработчикам обмениваться инструментами, устанавливать различные модули и управлять их зависимостями~\cite{npm}.

npm предоставляет возможность добавлять в свои проекты пакеты кода, а так же делиться своими, что избавляет от повторного написания больших объемов кода; создавать общую базу кода при командной разработке; управлять зависимостями проектов.

Ввиду того, что различия в скорости обработки и отображения информации программами, написанными с использованием этих языков программирования, не высоки, а также ввиду обладания опыта разработки на JS, мною был выбран JS в качестве основного языка программирования.