\subsection{Raspberry Pi}

Raspberry Pi (RPI) --- это одноплатный компьютер, то есть компьютер, все основные компоненты которого располагаются на одной плате. Это устройство можно использовать в различных сферах, ввиду высокой энергоэффективности, компактности и наличия всех минимально необходимых интерфейсов. Центральный процессор основан на архитектуре ARM. Одним из главных преимуществ RPI является наличие контактов интерфейса ввода/вывода общего назначения (GPIO), с помощью которых можно взаимодействовать с внешними устройствами.

Системные характеристики Raspberry Pi модели 3 B~\cite{rpi-site}:

\begin{enumerate}
  \item Микроархитектура --- Cortex-A53 (ARM v8);
  \item Частота центрального процессора --- 1,2 ГГц;
  \item Количество ядер --- 4;
  \item Объём оперативной памяти --- 1 ГБ;
  \item Количество контактов GPIO --- 40;
  \item Количество USB портов --- 4;
  \item Коммуникации --- Ethernet, Wi-Fi, Bluetooth.
\end{enumerate}

Операционная система RPI располагается на внешней microSD-карте, на которую можно записать один из дистрибутивов Linux. Официально с помощью установщика Raspberry Pi Imager можно установить одну из следующих операционных систем:

\begin{enumerate}
  \item Raspberry Pi OS;
  \item Ubuntu;
  \item Manjaro;
  \item RISC OS.
\end{enumerate}

Raspberry Pi OS является операционной системой, разработанной Raspberry Pi Foundation. Эта система основана на Debian~\cite{rpi-site}. Одним из главных недостатков этой операционной системы является малая производительность при высокой загрузке системы. Является проприетарной операционной системой.

Ubuntu также является операционной системой, основанной на Debian, разрабатывается группой Canonical~\cite{ubuntu}. Главным недостатком данной системы также является малая производительность системы после ``чистой установки'', то есть сразу после установки системы,  долгое время запуска операционной системы и невозможность работы с GPIO. Является проприетарной операционной системой.

RISC OS является операционной системой, разработанной специально для компьютеров, использующих центральный процессор архитектуры ARM~\cite{risc-os}. Редко используется ввиду малой популярности. Является проприетарной операционной системой.

Manjaro является операционной системой, основанной на Arch Linux. Производительность при высокой нагрузке является достаточной, чтобы выполнять несколько требовательных к ресурсам системы задач. Является операционной системой с открытым исходным кодом~\cite{manjaro}. В дальнейшем я буду пользоваться этой операционной системой.

Одним из главных недостатков Raspberry Pi является ограничение битовой глубины звука в 8 бит.
