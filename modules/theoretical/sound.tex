\subsection{Обработка звука}

\subsubsection{Advanced Linux Sound Architecture, PCM, RAW}

Для обработки звука в Linux-системах применяются два вида звуковых подсистем:

\begin{enumerate}
  \item ALSA --- Advanced Linux Sound Architecture, Продвинутая звуковая архитектура Linux;
  \item OSS --- Open Sound System, Открытая звуковая система.
\end{enumerate}

OSS использовался в Linux ядре ветки 2.4. Но, из-за наличия закрытого кода и платной лицензии, что контрастирует с названием системы, был заменён в последующих версиях ядра на ALSA.

ALSA предоставляет функциональность аудио драйвера и цифрового интерфейса музыкальных инструментов в Linux. ALSA тесно связана с ядром Linux. ALSA — программный микшер, который эмулирует совместимость для других слоёв. Также предоставляет API для программистов и работает с низкой и стабильной задержкой~\cite{alsa-project}. ALSA обладает следующими ключевыми особенностями:

\begin{enumerate}
  \item Эффективная поддержка всех типов звуковых интерфейсов, от любительских до профессиональных многоканальных интерфейсов;
  \item Полностью модульные звуковые драйвера;
  \item Аппаратное микширование нескольких каналов;
  \item Полнодуплексная работа;
  \item Поддерживающие многопроцессорность и драйверы с потоковой безопасностью, thread-safe драйверы;
  \item Открытая библиотека для упрощения разработки программного обеспечения и обеспечения высокоуровневого функционала;
  \item Поддержка более старого OSS API, обеспечение бинарной совместимости для большинства OSS программ.
\end{enumerate}

Наиболее часто для оцифровки аналоговых звуковых сигналов используется импульсно-кодовая модуляция (англ. PCM)~\cite{alsa-arch}. Данные, поступающие на записывающее устройство, подключенное к цифровому интерфейсу, кодируются с помощью PCM на звуковой карте, и затем передаются на звуковую подсистему ALSA.

При использовании PCM, аналоговый передаваемый сигнал преобразуется в цифровую форму посредством трёх операций: дискретизации по времени, квантования по амплитуде и кодирования. Сначала фиксируется амплитуда сигнала через определённые промежутки времени и регистрируются полученные значения амплитуды в виде округлённых цифровых значений. Затем полученные данные кодируются и передаются в канал связи.

Данные, полученные каналом связи, ALSA может представлять в 4 форматах файлов:

\begin{enumerate}
  \item VOC --- содержит заголовок, звуковые данные в сжатом виде;
  \item WAV --- содержит заголовок, звуковые данные не в сжатом виде;
  \item RAW --- не содержит заголовок, звуковые данные не в сжатом виде;
  \item AU --- содержит заголовок и аннотацию, звуковые данные в сжатом виде.
\end{enumerate}

При обработке звуковых данных, наиболее удобно использовать формат RAW. В этом случае, данные сохраняются без сжатия и заголовков, данные представлены в виде PCM-кодов.

\subsubsection{Утилиты arecord, aplay}

arecord --- это утилита для командной строки, предназначенная для записи звуковых данных со звуковой подсистемы ALSA. Она поддерживает несколько форматов записи данных, которые совпадают с форматами, поддерживаемыми ALSA, и одновременную запись с нескольких устройств~\cite{arecord}.

aplay представляет собой обратную arecord утилиту. С помощью aplay можно воспроизводить заранее записанные звуковые дорожки. Утилита также поддерживает все форматы, поддерживаемые ALSA~\cite{aplay}.

При работе как с arecord, так и с aplay, необходимо указать~\cite{arecord, aplay}:

\begin{enumerate}
  \item Формат выходного/входного файла (по-умолчанию wav);
  \item Формат данных, или битовая глубина (по-умолчанию U8 --- беззнаковое 8-битное);
  \item Количество каналов (по-умолчанию 1);
  \item Устройство, с которого происходит запись, или на котором звуковая дорожка должна проигрываться (по умолчанию устройство, указанное в файле конфигурации ALSA);
  \item Частоту дискретизации (по-умолчанию 8 кГЦ);
  \item Файл для вывода/ввода данных.
\end{enumerate}

Полученные arecord звуковые дорожки, в случае формата файла RAW, частоте дискретизации 8 кГц, использовании одного канала и формате данных U8, представляют собой 8000 записей со значениями от 0 до 255 каждую секунду.

Если необходимо конвейерно использовать данные утилиты arecord, необходимо использовать перенаправление потоков. То есть перенаправить стандартный вывод утилиты arecord в стандартный вход следующей команды. Для этого вместо имени выходного файла необходимо использовать знак тире (`-'), и в следующей за ней команде так же использовать знак тире. Например, чтобы перенаправить данные с микрофона на наушники, используя утилиты arecord и aplay, необходимо использовать следующий скрипт:

\begin{lstlisting}[style=ES6, language=bash]
  #!/bin/bash
  arecord -t raw -f cd | aplay - -t raw -f cd
\end{lstlisting}
